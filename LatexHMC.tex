\title{Simulation using Hamiltonian Monte Carlo}
\author{
        Ho Chung Leon  Law \\
            \and
        Nathan Cunningham\\
}
\date{\today}

\documentclass[12pt]{article}

\begin{document}
\maketitle

\begin{abstract}
In this report we aim to implement Hamiltonian Monte Carlo in R code and compare its effectiveness with other Monte Carlo methods \ldots
\end{abstract}

\section{Introduction}
Hamiltonian Monte Carlo is a method for sampling from a distribution 
\subsection{Aim}
\subsection{Motivation}
\subsection{Metropolis-Hastings failings}

\section{Hamiltonian dynamics}
The Hamiltonian equations model the evolution of a particle in a frictionless system over time, $t$, given its momentum, $p$, and position, $q$. The Hamilton equation consists of the potential energy of the particle, $U(q)$, and the kinetic energy, $K(p)$, and in Hamiltonian Monte Carlo is usually written as
\begin{equation}
H(q,p) = U(q) + K(p)
\end{equation}
This system evolves according to the following differential equations:
\begin{equation}
\frac{dq_{i}}{dt} = \frac{\delta H}{\delta p_{i}}
\end{equation}
\begin{equation}
\frac{dp_{i}}{dt} = \frac{-\delta H}{\delta q_{i}}
\end{equation}



\subsection{Properties of Hamiltonian dynamics}
\paragraph{Reversibility}
\paragraph{Volume preservation}
\paragraph{Conservation of the Hamiltonian}
\paragraph{Symplecticness}
\subsection{The Leapfrog}
\begin{equation}
p_{i}(t+\epsilon/2) = p_{i}(t) - (\epsilon/2)\frac{\delta U}{\delta q_{i}}(q(t))
\end{equation}
\begin{equation}
q_{i}(t+\epsilon) = q_{i}(t) - (\epsilon)\frac{p_{i}(t+\epsilon/2)}{m_{i}}
\end{equation}
\begin{equation}
p_{i}(t+\epsilon) = p_{i}(t+\epsilon/2) - (\epsilon/2)\frac{\delta U}{\delta q_{i}}(q(t+\epsilon))
\end{equation}

\section{Algorithm}
\begin{equation}
P(x) = \frac{1}{Z}exp(-E(x)/T)
\end{equation}
\begin{equation}
P(q,p) = \frac{1}{Z}exp(-H(q,p)/T)
\end{equation}
\begin{equation}
P(q,p) = \frac{1}{Z}exp(-U(q)/T)exp(-K(p)/T)
\end{equation}
\begin{equation}
K(p) = \sum_{i=1}^{d} \frac{p_{i}^{2}}{2m_{i}}
\end{equation}
\begin{equation}
min\{1,exp(-H(q^{*},p^{*})+H(q,p))\} = min\{1, exp(-U(q^{*}+U(q)-K(p^{*}+K(p))\}
\end{equation}
\subsection{Explanation pseudo-code}

\section{Comparisons}
\subsection{2-d Gaussian vs MH}
\subsection{Awkward initialisations}
\subsection{Computation times}
\subsection{High-dimensionality}
\subsection{Choices of $\epsilon$ and L}

\section{No U-turn sampler (NUTS)}

\bibliographystyle{abbrv}
\bibliography{main}

\end{document}
