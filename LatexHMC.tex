
\documentclass{article}
\usepackage{cite}
\usepackage{bm}
\usepackage{cite}
\usepackage{amsmath, amssymb}
\usepackage{amsthm}
\usepackage{amsfonts}
\usepackage{mathrsfs}
\usepackage{graphicx}
\usepackage{listings}
\usepackage{mathtools}
\usepackage{sidecap}
\sidecaptionvpos{figure}{c}
\DeclareGraphicsExtensions{.pdf,.png,.jpg}
\usepackage[normal]{caption}
\usepackage{relsize}
\usepackage{pdflscape}
\usepackage{geometry}
\usepackage[T1]{fontenc}
\usepackage[nottoc,numbib]{tocbibind}
\usepackage[sort&compress,round,comma,authoryear,numbers]{natbib}
%\usepackage[round]{natbib}   % omit 'round' option if you prefer square brackets
\bibliographystyle{plainnat}


\newtheorem*{mydef}{Definition}
\newtheorem*{prop}{Proposition}
\newtheorem*{thm}{Theorem}
\usepackage{fancyhdr}
\pagestyle{fancy}


\usepackage{placeins}
\usepackage{array,booktabs}

\lstset{
language=Matlab, % choose the language of the code
% basicstyle=10pt, % the size of the fonts that are used for the code
numbers=left, % where to put the line-numbers
numberstyle=\footnotesize, % the size of the fonts that are used for the line-numbers
stepnumber=1, % the step between two line-numbers. If it is 1 each line will be numbered
numbersep=5pt, % how far the line-numbers are from the code
% backgroundcolor=\color{white}, % choose the background color. You must add \usepackage{color}
showspaces=false, % show spaces adding particular underscores
showstringspaces=false, % underline spaces within strings
showtabs=false, % show tabs within strings adding particular underscores
% frame=single, % adds a frame around the code
% tabsize=2, % sets default tabsize to 2 spaces
% captionpos=b, % sets the caption-position to bottom
breaklines=true, % sets automatic line breaking
breakatwhitespace=false, % sets if automatic breaks should only happen at whitespace
escapeinside={\%*}{*)} % if you want to add a comment within your code
}


\title{University Of Oxford \\ OxWaSP 2015\\
Simulation using Hamiltonian Monte Carlo}

\begin{document}

\rhead{University Of Oxford }
\maketitle
\pagebreak
\begin{abstract}
\noindent
This project aims to explore aspects of using Hamiltonian Monte Carlo, from experimental design, data preparation, modelling and hypothesis testing. One main focus will be on the estimation of delay in fMRI response in an event related design using the method proposed by \cite{liao2002estimating}. Two efficient and robust estimators of delay are proposed and their performance is compared using simulated and real experimental data. Through these estimators, it is found that although the method is fairly robust for a block design, it is very sensitive in an event related design, especially when the true dispersion varies more than expected. \\
\noindent
\end{abstract}

\renewcommand{\abstractname}{Acknowledgements}
\begin{abstract}
\noindent \end{abstract}
\pagebreak
 \tableofcontents
 
 \pagebreak
 \section{Introduction} 
 \section{Algorithm}
 \end {document}