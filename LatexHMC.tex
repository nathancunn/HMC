\title{Simulation using Hamiltonian Monte Carlo}
\author{
        Leon Law \\
            \and
        Nathan Cunningham\\
}
\date{\today}

\documentclass[12pt]{article}

\begin{document}
\maketitle

\begin{abstract}
In this report we aim to implement Hamiltonian Monte Carlo in R code and compare its effectiveness with other Monte Carlo methods \ldots
\end{abstract}

\section{Introduction}
Hamiltonian Monte Carlo is a method for sampling from a distribution 
\subsection{Aim}
\subsection{Motivation}
\subsection{Metropolis-Hastings failings}

\section{Hamiltonian dynamics}
The Hamiltonian equations model the evolution of a particle in a frictionless system over time, $t$, given its momentum, $q$, and position, $p$. The Hamilton equation consists of the potential energy of the particle, $U(q)$, and the kinetic energy, $K(p)$, and in Hamiltonian Monte Carlo is usually written as
\begin{center}$H(q,p) = U(q) + K(p)$ \\ \end{center}
This system evolves according to the following differential equations:
\begin{center}
$\frac{dq_{i}}{dt} = \frac{\delta H}{\delta p_{i}}$ \\
$\frac{dp_{i}}{dt} = \frac{-\delta H}{\delta q_{i}}$
\end{center}
\subsection{Properties of Hamiltonian dynamics}
\subsection{The Leapfrog}


\section{Algorithm}
\subsection{Explanation pseudo-code}

\section{Comparisons}
\subsection{2-d Gaussian vs MH}
\subsection{Awkward initialisations}
\subsection{Computation times}
\subsection{High-dimensionality}
\subsection{Choices of $\epsilon$ and L}

\section{No U-turn sampler (NUTS)}

\bibliographystyle{abbrv}
\bibliography{main}

\end{document}
