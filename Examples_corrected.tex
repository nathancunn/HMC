\documentclass[10pt,a4paper,portrait]{article}
\usepackage[utf8]{inputenc}
\usepackage{amsmath}
\usepackage{amsfonts}
\usepackage{amssymb}
\usepackage{listings}
\usepackage{xcolor}
\definecolor{light-gray}{gray}{0.95}
\title{Simulating random samples using Hamiltonian Monte Carlo and the abcHMC package}
\begin{document}
\maketitle
\section{The HMC function}
The function HMC generates a random sample from a specified distribution. It
takes as arguments: total.samples, the number of samples to be simulated after
burnin; q.density, the density the samples should be drawn from; M, the mass
matrix; q, the starting values for the simulation; epsilon, the stepsize to be used in the leapfrog; L, the number of leapfrog steps to be made; diff.density, the derivative of the density to be drawn from; burnin the number of samples at
the beginning of the simulation to be discarded.
The following code will give a random sample of size 1,000 from a univariate
standard normal distribution using a step-size ($\epsilon$) of 0.05 and 20 leapfrog iterations, L.
\begin{lstlisting}[backgroundcolor = \color{lightgray},
                   language = R,
                   framexleftmargin = 1em]
HMC(total.samples = 10000, 
    q.density = function(x) dnorm(x,0,1), 
    M=1, 
    q = 0, 
    epsilon = 0.05, 
    L = 20, 
    diff.density = function(x) x, 
    burnin = 100)
\end{lstlisting}

This can be extended to the bivariate Gaussian by specifying a bivariate density
function. The following will simulate a bivariate Gaussian distribution with
highly-correlated covariates.

\begin{lstlisting}[backgroundcolor = \color{lightgray},
                   language = R,
                   framexleftmargin = 1em]
bivariate.density <- function(l) {
  dmvnorm(l, c(0, 0),matrix(c(1, 0.95, 0.95, 1), 2, 2))
}
bivariate.diff <- function(x) {
  solve(matrix(c(1, 0.95, 0.95, 1), 2, 2))%*%as.matrix(x)
}
HMC(total.samples = 10000, 
    q.density = bivariate.density, 
    q = c(-2,-2), 
    M = diag(2), 
    epsilon = 0.18, 
    L = 20, 
    diff.density = bivariate.diff, 
    burnin = 0)
\end{lstlisting}

Sampling from higher higher dimensions can be done in a similar manner. The
following will simulate 500 samples from a 150-dimensional distribution with
independent covariates.
\begin{lstlisting}[backgroundcolor = \color{lightgray},
                   language = R,
                   framexleftmargin = 1em]
multi.density <- function(l) {
  dmvnorm(l,rep(0,150),diag(seq(from=0.02,to=1,length=150)^2))
}
multi.diff <- function(x) {
  solve(diag(seq(from=0.02,to=1,length=150)^2))%*%as.matrix(x)
}
out.multidimension <- HMC(total.samples = 500, 
                          q.density = multi.density, 
                          q = rep(0,150), 
                          M=diag(150), 
                          epsilon = 0.014, 
                          L = 100, 
                          diff.density = multi.diff, 
                          burnin = 0)
\end{lstlisting}
\section{The WordPrint function}
The function $WordPrint$ is used to plot a random sample from a distribution
whose probability density resembles a chosen word. It takes two arguments,
word: the word you would like plotted, given as a character string; and samples:
the number of simulated points to be used in the plot. Before running this function the dataset specifying the underlying models must be loaded.

\begin{lstlisting}[backgroundcolor = \color{lightgray},
                   language = R,
                   framexleftmargin = 1em]
letter.models <- data("letter.models")
WordPrint(word = "abcHMC", samples = 2500)
\end{lstlisting}

\end{document}
